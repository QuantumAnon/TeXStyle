% \iffalse meta-comment
%
% MIT License
%
% Copyright (c) 2017 Gautam Chaudhuri
%
% Permission is hereby granted, free of charge, to any person obtaining a copy
% of this software and associated documentation files (the "Software"), to deal
% in the Software without restriction, including without limitation the rights
% to use, copy, modify, merge, publish, distribute, sublicense, and/or sell
% copies of the Software, and to permit persons to whom the Software is
% furnished to do so, subject to the following conditions:
%
% The above copyright notice and this permission notice shall be included in all
% copies or substantial portions of the Software.
%
% THE SOFTWARE IS PROVIDED "AS IS", WITHOUT WARRANTY OF ANY KIND, EXPRESS OR
% IMPLIED, INCLUDING BUT NOT LIMITED TO THE WARRANTIES OF MERCHANTABILITY,
% FITNESS FOR A PARTICULAR PURPOSE AND NONINFRINGEMENT. IN NO EVENT SHALL THE
% AUTHORS OR COPYRIGHT HOLDERS BE LIABLE FOR ANY CLAIM, DAMAGES OR OTHER
% LIABILITY, WHETHER IN AN ACTION OF CONTRACT, TORT OR OTHERWISE, ARISING FROM,
% OUT OF OR IN CONNECTION WITH THE SOFTWARE OR THE USE OR OTHER DEALINGS IN THE
% SOFTWARE.
%
% \fi
% \iffalse
%<package>\NeedsTeXFormat{LaTeX2e}
%<package>\ProvidesPackage{TeXStyle}
% \fi
%
%   \subsection{TeXStyle options}
%     This section covers the options available on loading the TeXStyle package.
%     Most of these are implemented because other packages need them.
%     Two options to note are the |useHyperrefMacros| and its complement  the
%     |noHyperrefMacros| options.
%     If the package |hyperref| is loaded before |TeXStyle|, the corresponding
%     boolean is automatically set to true and certain macros are then defined
%     to make referencing slightly easier.
%     To override this behaviour, specify |noHyperrefMacros|.
%     Key-value options are implemented using the |kvoptions| package.
% \iffalse
%<*package>
% \fi
%    \begin{macrocode}
  \RequirePackage{kvoptions}
  \SetupKeyvalOptions{
    family=TS,
    prefix=TS@
  }
  \DeclareBoolOption[false]{draft}
  \DeclareBoolOption[false]{defaultFonts}
  \DeclareBoolOption[false]{HyperrefMacros}
  \@ifpackageloaded{hyperref}{
    \TS@HyperrefMacrostrue
  }
  {
    \ifTS@HyperrefMacros
    \PackageWarning{TeXStyle}{
      Loading hyperref macros even though hyperref has
      not been detected yet.
      If hyperref has been loaded later in the document, ignore this message.
    }
    \fi
  }
  \DeclareStringOption[section]{envRefreshLevel}
  \DeclareDefaultOption{
    \PackageWarning{TeXStyle}{Unknown Option '\CurrentOption'}
    }
    \ProcessKeyvalOptions*

%    \end{macrocode}
% \iffalse
%</package>
% \fi
%
%   \subsection{Colours}
%
%     The following colours are named to provide some consistency across the
%     style. The required dependencies are:
%     \begin{itemize}
%       \item |color|: Control over the colour of various elements.
%       \item |xcolor|: A driver-independent implementation of color.
%     \end{itemize}
%
% \iffalse
%<*package>
% \fi
%    \begin{macrocode}
\RequirePackage{color}
\RequirePackage{xcolor}

%    \end{macrocode}
% \iffalse
%</package>
% \fi
%
% \iffalse
%<*package>
% \fi
%    \begin{macrocode}

  \definecolorset{HTML}{accent-}{}{%
    red,EF2D56;%
    green,7DDF64;%
    blue,5BC0EB%
  }

  \definecolorset{HTML}{line-}{}{%
    blue,1D2951;%
    grey,90909A;%
    black,02111B%
  }

  \definecolorset{HTML}{back-}{}{%
    grey,403F4C;%
    blue,C6E0FF%
  }

  \definecolorset{HTML}{ts-}{}{%
    0,FFFFEE;%
    1,F1F1F0;%
    2,F0F0EE;%
    3,F2E7DA;%
    4,F2BC79;%
    5,C2D2E9;%
    6,5E83BA;%
    7,8899AA;%
    8,3A4E7A;%
    9,004AA8;%
    10,114488;%
    11,003366;%
    12,2B2B2B;%
    13,091D36;%
    14,221100;%
    15,111110%
  }
%    \end{macrocode}
% \iffalse
%<*package>
% \fi
%
%
%     \newcommand\crule[1][black]{\textcolor{#1}{\rule{1em}{1em}}}
%     \noindent
%     The colours are:
%
%     \newcounter{i}
%     \setcounter{i}{0}
%     \loop
%       \ifnum \value{i} < 16
%       \noindent
%       ts-\thei : \crule[ts-\thei],\\
%       \stepcounter{i}
%     \repeat
%     accent-red : \crule[accent-red],\\
%     accent-green : \crule[accent-green],\\
%     accent-blue : \crule[accent-blue],\\
%     line-blue : \crule[line-blue],\\
%     line-grey : \crule[line-grey],\\
%     line-black : \crule[line-black],\\
%     back-grey : \crule[back-grey],\\
%     back-blue : \crule[back-blue]