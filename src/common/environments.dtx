%     The environments defined here are intended to be used to clearly delineate
%     various parts of a document.
%     It is recommended that the sub<X> environments are used within
%     their respective parent group.

%     \subsection{Dependencies}
%       The following packages are required in order to generate and style the
%       environments.
%       \begin{itemize}
%         \item |amsthm| provides the base for generating and configuring
%           theorem environments.
%         \item |mdframed| with the |framemethod=tikz| option allows theorem
%           environments to be framed with lines.
%         \item |listings| provides environments for including source code.
%         \item |enumitem| with the |shortlabels| option allows short specifiers
%           for the bullets on enumerated lists.
%         \item \textit{(optional)}|aliascnt| is loaded if the
%           |useHyperrefMacros| option is specified.
%           It provides aliased counters to help hyperref to properly reference
%           theorems, etc.
%       \end{itemize}
%
%       In addition, we use the colours defined in |src/common/colours| so
%       this must be included as a dependency when generating the file.
% \iffalse
%<*package>
% \fi
%    \begin{macrocode}
%%------------------------------------------------------------
%% Environments
%% |- Dependencies
  \RequirePackage{amsthm}
  \RequirePackage[framemethod=tikz]{mdframed}
  \RequirePackage{listings}
  \RequirePackage[shortlabels]{enumitem}

  \ifTS@HyperrefMacros
  \RequirePackage{aliascnt}
  \fi

%    \end{macrocode}
% \iffalse
%</package>
% \fi
%
%     \subsection{Rough Theorem Styles}
%
%         These styles are used to roughly define how each theorem should look.
%         They are then further refined for each individual type.
%         The arguments are:
%
% \iffalse
%<*package>
% \fi
%    \begin{macrocode}
%% |- Rough theorem styles
%%     No break between heading and body, numbered
  \newtheoremstyle{NoBreak}
    {}% measure of space to leave above the theorem.
    {}% measure of space to leave below the theorem.
    {\itshape}% name of font to use in the body of the theorem
    {}% measure of space to indent
    {}% name of head font
    {}% punctuation between head and body
    {.5em}% space after theorem head
    %; " " = normal inter-word space
    % Manually specify theorem head
    {\thmname{#1} \thmnumber{#2}: \thmnote{#3}}

%    \end{macrocode}
% \iffalse
%</package>
% \fi
%       The other styles are defined similarly
% \iffalse
%<*package>
% \fi
%    \begin{macrocode}
%% No break between heading and body, not numbered
  \newtheoremstyle{NoBreakNoNumber}{}{}{\itshape}% chktex 6
    {}{}{}{.5em}{\thmname{#1}: \thmnote{#3}}

%% Break between heading and body, numbered and bold
  \newtheoremstyle{BreakBold}{}{}{\itshape}% chktex 6
    {}{\bfseries}{}{\newline}%
    {\thmname{#1} \thmnumber{#2}: \thmnote{#3}}

%% Break between heading and body, numbered
  \newtheoremstyle{Break}{}{}{\itshape}%
    {}{}{}{\newline}%
    {\thmname{#1} \thmnumber{#2}: \thmnote{#3}}

%% Break between heading and body, not numbered
  \newtheoremstyle{BreakNoNumber}{}{}{\itshape}%
    {}{}{}{\newline}%
    {\thmname{#1}: \thmnote{#3}}

%% Break between heading and body, bold heading,
%% not numbered
  \newtheoremstyle{BreakBoldNoNumber}{}{}{\itshape}%
    {}{\bfseries}{}{\newline}%
    {\thmname{#1}: \thmnote{#3}}

%% No break between heading and body,
%% bold heading, numbered
  \newtheoremstyle{NoBreakBold}{}{}{\itshape}% chktex 6
    {}{\bfseries}{}{.5em}%
    {\thmname{#1} \thmnumber{#2}: \thmnote{#3}}

%% No break between heading and body, bold heading,
%% not numbered
  \newtheoremstyle{NoBreakBoldNoNumber}{}{}{\itshape}%
    {}{\bfseries}{}{.5em}{\thmname{#1}: \thmnote{#3}}

%    \end{macrocode}
% \iffalse
%</package>
% \fi
%       We also define the following lengths.
% \iffalse
%<*package>
% \fi
%       \begin{macrocode}
%% |- Length definitions
%%    |- Declare lengths
  \newlength{\envVertSkip}
  \newlength{\envTopInner}
  \newlength{\envBotInner}
  \newlength{\envLeftSkip}
  \newlength{\subLeftSkip}
  \newlength{\subTopSkip}
  \newlength{\subBotSkip}
  \newlength{\headerOffset}

%%    |- Define lengths
  \setlength{\envVertSkip}{.5cm}
  \setlength{\envTopInner}{.1cm}
  \setlength{\envBotInner}{.25cm}
  \setlength{\envLeftSkip}{.5cm}
  \setlength{\subLeftSkip}{.1cm}
  \setlength{\subTopSkip}{0cm}
  \setlength{\subBotSkip}{.1cm}
  \setlength{\headerOffset}{-.3cm}

%       \end{macrocode}
% \iffalse
%</package>
% \fi
%
% Additionally, a filled black square
% marks the end of a proof as opposed
%       to an empty one.
%
% \iffalse
%<*package>
% \fi
%    \begin{macrocode}
  \renewcommand{\qedsymbol}{$\blacksquare$}

%    \end{macrocode}
% \iffalse
%</package>
% \fi
%
%     \subsection{Detailed Theorem Environments}
%       The following |amsthm| environments are defined within TeXStyle.
%
%       \subsubsection{Hyperref Macros}
%       An additional counter is defined if the boolean |HyperrefMacros| is
%       true, this follows the numbering of the parent counter while simulating
%       another so that the autoref macro in |hyperref| works properly.
%       If hyperref is loaded, referencing should look something like this:
%       \begin{itemize}
%         \item \autoref{thm:First} goes to the first Theorem.
%         \item \autoref{pro:First}, the first Proposition and
%         \autoref{sPro:First}, the first Sub-Proposition.
%         \item \autoref{def:First}, the first Definition and
%         \autoref{sDef:First}, the first Sub-Definition.
%         \item \autoref{exa:First}, the first Example and
%         \autoref{sExa:First}, the first Sub-Example.
%         \item \autoref{lem:First}, the first Lemma.
%         \item \autoref{cor:First}, the first Corollary.
%         \item \autoref{rem:First}, the first Remark.
%       \end{itemize}
%       The numbering of the theorems is set to reset by default when the
%       section resets.
%       This behaviour can be changed by using the |envRefreshLevel| option.
%       
%
% \begin{environment}{theorem}
% \iffalse
%<*package>
% \fi
%    \begin{macrocode}
  \newtheoremstyle{Theorem}{}{}{}%
    {}{\bfseries}{}{\newline}%
    {\thmname{#1} \thmnumber{#2}: \thmnote{#3}}

  \mdfdefinestyle{theorem}{%
    skipabove=\envVertSkip,%
    skipbelow=\envVertSkip,%
    innertopmargin=\envTopInner,%
    innerbottommargin=\envBotInner,%
    innerleftmargin=\envLeftSkip,%
    linecolor=line-blue,%
    linewidth=.05cm,%
    backgroundcolor=back-blue!20%
    }

  \theoremstyle{Theorem}

  \newmdtheoremenv[style=theorem]{theorem}%
  {Theorem}[\TS@envRefreshLevel]

  \newtheoremstyle{Theorem*}{}{}{}%
    {}{\bfseries}{}{\newline}%
    {\thmname{#1}:\thmnote{\ #3}}%
  \theoremstyle{Theorem*}
  \newmdtheoremenv[style=theorem]{theorem*}{Theorem}

%    \end{macrocode}
% \iffalse
%</package>%
% \fi
%
% \end{environment}
%
% \begin{environment}{proposition}
% \iffalse
%<*package>
% \fi
%    \begin{macrocode}
  \newtheoremstyle{Proposition}{}{}{\itshape}%
    {}{\bfseries}{}{\newline}%
    {\thmname{#1} \thmnumber{#2}: \thmnote{#3}}

  \newtheoremstyle{Proposition*}{}{}{\itshape}%
  {}{\bfseries}{}{\newline}%
  {\thmname{#1}: \thmnote{#3}}

  \mdfdefinestyle{proposition}{%
    skipabove=\envVertSkip,%
    skipbelow=\envVertSkip,%
    innertopmargin=\envTopInner,%
    innerbottommargin=\envBotInner,%
    innerleftmargin=\envLeftSkip,%
    backgroundcolor=back-grey!5,%
    linecolor=line-grey,%
    theoremtitlefont=\sffamily\bfseries%
  }

  \mdfdefinestyle{subPro}{%
    skipabove=\subTopSkip,%
    skipbelow=\subBotSkip,%
    innertopmargin=\headerOffset,%
    innerbottommargin=0cm,%
    innerleftmargin=\subLeftSkip,%
    backgroundcolor=back-grey!5,%
    hidealllines=true,%
    leftline=true,%
    linecolor=line-grey,%
    linewidth=.05cm%
  }
  \ifTS@HyperrefMacros
    \newaliascnt{proposition}{theorem}
    \theoremstyle{Proposition}
    \newmdtheoremenv[style=proposition]{proposition}%
    [proposition]{Proposition}
    \aliascntresetthe{proposition}

    % Refreshes every proposition
    \theoremstyle{NoBreakBold}
    \newmdtheoremenv[style=subPro]{subPro}{}[proposition]

    % autoref commands
    \providecommand*{\propositionautorefname}{Proposition}
    \providecommand*{\subProautorefname}{Proposition}

  \else
    \theoremstyle{Proposition}
    \newmdtheoremenv[style=proposition]{proposition}%
    [theorem]{Proposition}

    % Refreshes counter every proposition
    \theoremstyle{NoBreakBold}
    \newmdtheoremenv[style=subPro]{subPro}{}[theorem]
  \fi

  \theoremstyle{Proposition*}
  \newmdtheoremenv[style=proposition]{proposition*}%
  {Proposition}

%    \end{macrocode}
% \iffalse
%</package>
% \fi
%
% \end{environment}
%
% \begin{environment}{definition}
% \iffalse
%<*package>
% \fi
%    \begin{macrocode}
  \newtheoremstyle{Definition}{5pt}{5pt}{\sffamily}%
    {}{\sffamily\bfseries}{}{\newline}%
    {\thmname{#1}\ \thmnumber{#2}:\ \thmnote{#3}}

  \newtheoremstyle{Definition*}{5pt}{5pt}{\sffamily}%
    {}{\sffamily\bfseries}{}{\newline}%
    {\ \thmname{#1}:\thmnote{#3}}

  \mdfdefinestyle{definition}{%
    skipabove=\envVertSkip,%
    skipbelow=\envVertSkip,%
    innertopmargin=\envTopInner,%
    innerbottommargin=\envBotInner,%
    innerleftmargin=\envLeftSkip,%
    linecolor=line-black,%
    linewidth=.05cm,%
    nobreak%
    }

  \newtheoremstyle{SubDefinition}{}{}{\sffamily}%
    {}{\bfseries}{}{.5em}%
    {\thmname{#1}\ \thmnumber{#2}:\ \thmnote{#3}}

  \mdfdefinestyle{subDef}
  {%
    % skipabove needs to be a bit
    % higher to balance with the lower skip
    skipabove=\dimexpr(\subTopSkip + .25cm), % chktex 1
    skipbelow=\subBotSkip,
    innertopmargin=0cm,
    % =\dimexpr(\headerOffset + .3cm)
    % we want subDefs to be flush with the main definition
    innerleftmargin=0cm,
    hidealllines=true,
    topline=true,
    linecolor=line-black,
    linewidth=.01cm
  }
  \ifTS@HyperrefMacros
      \theoremstyle{Definition}

    \newaliascnt{definition}{theorem}
    \newmdtheoremenv[style=definition]{definition}%
    [definition]{Definition}

    \aliascntresetthe{definition}
    \theoremstyle{SubDefinition}

    % Refreshes every definition
    \newmdtheoremenv[style=subDef]{subDef}{}[definition]

    \providecommand*{\definitionautorefname}{Definition}
    \providecommand*{\subDefautorefname}{Definition}

  \else
    \theoremstyle{Definition}

    \newmdtheoremenv[style=definition]{definition}%
    [theorem]{Definition}

    \theoremstyle{SubDefinition}
    % Refreshes every definition
    \newmdtheoremenv[style=subDef]{subDef}{}[theorem]
  \fi

    \theoremstyle{Definition*}

    \newmdtheoremenv[style=definition]{definition*}%
    {Definition}

%    \end{macrocode}
% \iffalse
%</package>
% \fi
%
% \end{environment}
%
% \begin{environment}{example}
% \iffalse
%<*package>
% \fi
%    \begin{macrocode}
  \mdfdefinestyle{example}{%
    skipabove=\envVertSkip, skipbelow=\envVertSkip,%
    % want the top and bottom of the example to be
% flush with the bar
    innertopmargin=\headerOffset,%
    innerbottommargin=0cm,%
    innerleftmargin=\envLeftSkip,%
    hidealllines=true,%
    leftline=true,%
    linecolor=line-black,%
    linewidth=.1cm%
    }

  \mdfdefinestyle{subExa}{%
    skipabove=\subTopSkip,%
    skipbelow=\subTopSkip,%
    innertopmargin=\headerOffset, % flush with bar
    innerbottommargin=0cm, % ditto
    innerleftmargin=\subLeftSkip,%
    hidealllines=true,%
    leftline=true,%
    linecolor=line-black,%
    linewidth=.01cm%
  }

  \ifTS@HyperrefMacros
    \theoremstyle{BreakBold}

    \newaliascnt{example}{theorem}
    \newmdtheoremenv[style=example]{example}[example]{Example}
    \aliascntresetthe{example}
  \else
    \theoremstyle{BreakBold}

    \newmdtheoremenv[style=example]{example}[theorem]{Example}
  \fi

  \theoremstyle{NoBreakNoNumber}
  \newmdtheoremenv[style=subExa]{subExa}{}

  \theoremstyle{BreakBoldNoNumber}
  \newmdtheoremenv[style=example]{example*}{Example}

%    \end{macrocode}
% \iffalse
%</package>
% \fi
%
% \end{environment}
%
% \begin{environment}{exercise}
% \iffalse
%<*package>
% \fi
%    \begin{macrocode}
  \mdfdefinestyle{exercise}{%
    skipabove=\envVertSkip, skipbelow=\envVertSkip,%
    innertopmargin=\headerOffset,%
    innerbottommargin=0cm,%
    innerleftmargin=\envLeftSkip,%
    hidealllines=true,%
    leftline=true,%
    rightline=true,%
    linecolor=line-black,%
    linewidth=.1cm%
    }

  \mdfdefinestyle{subExr}{%
    skipabove=\subTopSkip, skipbelow=\subTopSkip,%
    innertopmargin=\dimexpr(\headerOffset + .2cm),% chktex 1
    innerbottommargin=.25cm,%
    innerleftmargin=\subLeftSkip,%
    hidealllines=true,%
    topline=true,%
    bottomline=true,%
    linecolor=line-black,%
    linewidth=.01cm%
  }

  \theoremstyle{BreakBoldNoNumber}
  \newmdtheoremenv[style=exercise]{exercise}{Exercise}

  \theoremstyle{NoBreak}
  \newmdtheoremenv[style=subExr]{subExr}{}

%    \end{macrocode}
% \iffalse
%</package>
% \fi
%
% \end{environment}
%
% \begin{environment}{lemma}
% \iffalse
%<*package>
% \fi
%    \begin{macrocode}
  \mdfdefinestyle{lemma}{%
    skipabove=\envVertSkip, skipbelow=\envVertSkip,%
    innertopmargin=\headerOffset,%
    innerbottommargin=0cm, % flush
    innerleftmargin=\envLeftSkip,%
    hidealllines=true,%
    leftline=true, linecolor=line-grey,%
    linewidth=.05cm%
  }
  \ifTS@HyperrefMacros
    \theoremstyle{Break}

    \newaliascnt{lemma}{theorem}

    \newmdtheoremenv[style=lemma]{lemma}[lemma]{Lemma}

    \aliascntresetthe{lemma}

    \providecommand*{\lemmaautorefname}{Lemma}
  \else
    \theoremstyle{Break}
    \newmdtheoremenv[style=lemma]{lemma}[theorem]{Lemma}
  \fi

    \theoremstyle{BreakNoNumber}

    \newmdtheoremenv[style=lemma]{lemma*}{Lemma}

%    \end{macrocode}
% \iffalse
%</package>
% \fi
%
% \end{environment}
%
% \begin{environment}{corollary}
% \iffalse
%<*package>
% \fi
%    \begin{macrocode}
  \mdfdefinestyle{corollary}{%
    skipabove=\envVertSkip, skipbelow=\envVertSkip,%
    innertopmargin=\headerOffset, innerbottommargin=0cm,%
    innerleftmargin=\envLeftSkip,%
    hidealllines=true,%
    leftline=true,%
    linecolor=accent-blue,%
    linewidth=.05cm%
    }
    \ifTS@HyperrefMacros
      \theoremstyle{BreakBold}

      \newaliascnt{corollary}{theorem}
      
      \newmdtheoremenv[style=corollary]{corollary}%
      [corollary]{Corollary}

      \aliascntresetthe{corollary}

      \providecommand*{\corollaryautorefname}{Corollary}
    \else
      \theoremstyle{BreakBold}
      \newmdtheoremenv[style=corollary]{corollary}%
        [theorem]{Corollary}
    \fi

      \theoremstyle{BreakBoldNoNumber}
      \newmdtheoremenv[style=corollary]{corollary*}%
        {Corollary}

%    \end{macrocode}
% \iffalse
%</package>
% \fi
%
% \end{environment}
%
% \begin{environment}{remark}
% \iffalse
%<*package>
% \fi
%    \begin{macrocode}
  \mdfdefinestyle{remark}{%
    skipabove=\envVertSkip, skipbelow=\envVertSkip,%
    innertopmargin=\headerOffset, innerbottommargin=0cm,%
    innerleftmargin=\envLeftSkip,%
    hidealllines=true,%
    leftline=true,%
    linecolor=line-black%
    }

  \ifTS@HyperrefMacros
    \theoremstyle{BreakBold}

    \newaliascnt{remark}{theorem}

    \newmdtheoremenv[style=remark]{remark}[remark]{Remark}

    \aliascntresetthe{remark}

    \providecommand*{\remarkautorefname}{Remark}
  \else
    \theoremstyle{BreakBold}
    \newmdtheoremenv[style=remark]{remark}{Remark}
  \fi

    \theoremstyle{BreakBoldNoNumber}
    \newmdtheoremenv[style=remark]{remark*}{Remark}

%    \end{macrocode}
% \iffalse
%</package>
% \fi
%
% \end{environment}
%
% \begin{environment}{note}
% \iffalse
%<*package>
% \fi
%    \begin{macrocode}
  \theoremstyle{NoBreakBoldNoNumber}
  \makeatletter
  \@ifclassloaded{beamer}
  {}
  {\newtheorem{note}{Note}}
  \makeatother

%    \end{macrocode}
% \iffalse
%</package>
% \fi
%
% \end{environment}
%
%   \subsubsection{Examples of Theorem Environments}
%   Here are a few examples of how the individual theorem environments look.
%
%   \clearpage
%   \lipsum[1]
%   \begin{theorem}[Theorem Title]\label{thm:First}
%     This is a theorem, this is the theorem sub-title
%
%     $\forall x \in S, x \in S$,
%     normal text looks like this.
%     \lipsum[1]
%
%     This is a centered equation
%     \[a + b = c + d\]
%
%     \lipsum[2]
%   \end{theorem}
%   \begin{proof}
%     This is a proof, this is the first line of the proof.
%
%     This is now a multi-line proof, this is the second line of the proof
%     with a hard break in between.
%     This is a soft line break.
%     \lipsum[1]
%
%     This is a hard line break.
%   \end{proof}
%
%   \begin{proof}[Alternative proof title]
%     This is a proof with an alternate title.
%   \end{proof}
%   \lipsum[2]
%   \begin{proposition}[Proposition Title]\label{pro:First}
%     This is a proposition, this is the proposition sub-title.
%
%     Let $S \subset \RR$.
%     This is what normal text looks like.
%     \lipsum[3]
%
%     \begin{subPro}\label{sPro:First}
%       This is a sub-proposition, this is the sub-proposition title.
%
%       $\forall x \in S, x \in S$,
%       this is how normal text looks.
%       \lipsum[4]
%
%     \end{subPro}
%     \begin{subPro}\label{sPro:Second}
%       This is a sub-proposition, this is the sub-proposition title.
%
%       $\forall x \in S, x \in S$,
%       this is how normal text looks.
%       \lipsum[5]
%
%     \end{subPro}
%   \end{proposition}
%   \lipsum[6]
%   \begin{definition}[Definition Title]\label{def:First}
%     This is a definition, this is the definition sub-title.
%
%     Let $S \subset \RR$.
%     This is what normal text looks like.
%     \lipsum[7]
%
%     \begin{subDef}\label{sDef:First}
%       This is a sub-definition, this is the sub-definition title.
%
%       $\forall x \in S, x \in S$,
%       this is how normal text looks.
%       \lipsum[8]
%     \end{subDef}
%     \begin{subDef}\label{sDef:Second}
%       This is a sub-definition, this is the sub-definition title.
%
%       $\forall x \in S, x \in S$,
%       this is how normal text looks.
%       \lipsum[9]
%     \end{subDef}
%   \end{definition}
%   \lipsum[10]
%   \begin{example}[Example Title]\label{exa:First}
%     This is an example, this is the example sub-title.
%
%     Let $S \subset \RR$.
%     This is what normal text looks like.
%     \lipsum[11]
%
%     \begin{subExa}\label{sExa:First}
%       This is a sub-example. This is the sub-example title.
%
%       $\forall x \in S, x \in S$,
%       this is how normal text looks.
%       \lipsum[12]
%
%     \end{subExa}
%     \begin{subExa}\label{sExa:Second}
%       This is a sub-example. This is the sub-example title.
%
%       $\forall x \in S, x \in S$,
%       this is how normal text looks.
%       \lipsum[13]
%
%     \end{subExa}
%   \end{example}
%   \lipsum[10]
%   \begin{exercise}[Exercise Title]\label{exr:First}
%     This is an exercise, this is the exercise sub-title.
%
%     Let $S \subset \RR$.
%     This is what normal text looks like.
%     \lipsum[11]
%
%     \begin{subExr}\label{sExr:First}
%       This is a sub-exercise. This is the sub-exercise title.
%
%       $\forall x \in S, x \in S$,
%       this is how normal text looks.
%       \lipsum[12]
%
%     \end{subExr}
%     \begin{subExr}\label{sExr:Second}
%       This is a sub-exercise. This is the sub-exercise title.
%
%       $\forall x \in S, x \in S$,
%       this is how normal text looks.
%       \lipsum[13]
%
%     \end{subExr}
%   \end{exercise}
%   \lipsum[1]
%   \begin{lemma}[Lemma title]\label{lem:First}
%     This is a Lemma, this is the lemma sub-title.
%
%     $\forall x \in S, x \in S$,
%     normal text looks like this.
%     \lipsum[13]
%   \end{lemma}
%   \lipsum[15]
%   \begin{corollary}[Corollary title]\label{cor:First}
%     This is a Corollary, this is the corollary sub-title.
%
%     $\forall x \in S, x \in S$,
%     normal text looks like this.
%     \lipsum[16]
%
%   \end{corollary}
%
%   \begin{remark}[Remark title]\label{rem:First}
%     This is a remark, this is the remark sub-title.
%
%     $\forall x \in S, x \in S$,
%     normal text looks like this.
%     \lipsum[17]
%
%   \end{remark}
%   \lipsum[18]
%   \begin{note}
%     This is a note.
%     \lipsum[19]
%   \end{note}
%
%   \subsection{Source Code Environments}
%     The listings environment is given in order to typeset source code for various languages.
%     For more information, read the documentation of the listings environment.
%
% \iffalse
%<*package>
% \fi
%    \begin{macrocode}
  \lstset{%
    basicstyle=\fontsize{8}{10}\ttfamily,
    aboveskip={0.8\baselineskip},
    belowskip={0.8\baselineskip},
    columns=fixed,
    extendedchars=true,
    breaklines=true,
    tabsize=4,
    prebreak=\raisebox{0ex}[0ex][0ex]%
    {\ensuremath{\hookleftarrow}},
    frame=lines,
    showtabs=true,
    showspaces=false,
    showstringspaces=false,
    keywordstyle=\color{line-blue},
    commentstyle=\color{accent-green},
    stringstyle=\color{accent-red},
    numbers=left,
    numberstyle=\small,
    stepnumber=1,
    numbersep=10pt,
    captionpos=t,
    numbers=left,
    stepnumber=1,
    firstnumber=1,
    numberfirstline=true
    }

%    \end{macrocode}
% \iffalse
%</package>
% \fi
%
%
%     Given below is an example of the output with the default options:
%     \lstinputlisting[language=Ruby,title=An example of the listings environment]{../common/example.rb}
